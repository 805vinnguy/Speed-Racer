% Speed Racer
%
% Vinh Nguyen
% October 17, 2018

\documentclass[12pt, oneside, a4paper]{article}

\usepackage{mathtools}

\begin{document}

\title{Speed Racer}
\author{Vinh Nguyen}
\date{October 17, 2018}
\maketitle
\newpage

\section*{Description}
	\paragraph{
		There is a speed racer who must rescue their friend at the top of Mount Domo,
		which is \emph{m} km away, as quickly as possible.  For a given speed 
		\emph{v} in km/hr, the amount of fuel \emph{t} consumed in L/hr is: 
		$$a*v^4 + b*v^3 + c*v^2 + d*v = t$$
	}
	\paragraph{
		What is the maximum speed, speed racer can drive to reach the top of Mount
		Domo without running out of gas?
	}
	
\section*{Solution}
	\paragraph{
		From the equation, $a*v^4 + b*v^3 + c*v^2 + d*v = t$, solving for \emph{v}
		will yield the maximum speed possible given \emph{t} fuel, which is the
		solution if $t <= m$.
		\hspace{0pt}
		\emph{m} is the factor which will determine maximum speed.  If we take the
		equation and factor in \emph{m}, following correct unit conventions we should
		arrive at our answer.
		$$L/hr = L/km * km/hr$$
		We simply solve: $$a*v^4 + b*v^3 + c*v^2 + d*v = t/m * v$$
						 $$a*v^3 + b*v^2 + c*v + d = t/m$$
		for \emph{v}.
	}

\section*{Input}
	\paragraph{
		An input file containing a problem per line.\hspace{0pt}
		Each line containing 6 single-space separated positive floating point values:
		a b c d m t
		No input value will exceed 1000.  
		There will always be a solution.
		Truncate, rather than round, the final result.
	}	

\section*{Output}
	\paragraph{
		A single floating point value representing the maximum speed, speed racer can
		travel to reach the top of Mount Domo without running out of fuel, formatted
		as a decimal with exactly two digits right of the decimal point and no leading
		zeros.
	}
	
\end{document}